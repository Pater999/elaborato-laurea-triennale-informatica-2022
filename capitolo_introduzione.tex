\chapter{Introduzione}
\label{cha:introduzione}

L'obiettivo di questo progetto è quello di semplificare la ricerca di informazioni sul trasporto pubblico trentino mediante la progettazione e implementazione di un Chatbot. Infatti, al momento, a livello territoriale esistono diverse applicazioni che consentono di controllare orari, percorsi, prezzi, disponibilità, ma non è ancora presente un unico applicativo che integri tutte le informazioni utili. La necessità di un software completo e intuitivo è supportata anche dalla diffusione della \textit{smart mobility}, che consente, attraverso la tecnologia, di migliorare diversi ambiti connessi al trasporto.


Per poter affrontare al meglio le fasi di analisi e progettazione sono risultate fondamentali le conoscenze acquisite con il corso di ingegneria del software, soprattutto riguardo all'analisi dei requisiti e le varie metodologie di progettazione. Mentre, in riferimento a \textit{user experience} e \textit{user interface} mi ha aiutato molto il corso di Human-Computer Interaction, grazie al quale ho potuto comprendere l'importanza del feedback degli utenti in tutte le fasi e, attraverso le metodologie di valutazione apprese, ho svolto una fase di evaluation con gli user test. Seppur con un ruolo minore, anche i corsi di programmazione e database sono risultati utili per l'implementazione di questo progetto. Infatti, come vedremo successivamente, le tecnologie e i linguaggi utilizzati per l'implementazione del progetto non sono stati trattati durante i corsi citati in precedenza ma essi sono stati una buona base di partenza. Per imparare a lavorare con queste nuove tecnologie mi sono informato attraverso la documentazione ufficiale, video tutorial su \textit{Youtube} e integrando con ciò che ho appreso durante il tirocinio formativo presso l'azienda \textit{Zupit}. 

All'interno dei vari capitoli andrò quindi ad analizzare, in primis, i concetti di \textit{smart city} e \textit{smart mobility}, entro cui è possibile valutare l'utilità di questo progetto, e i competitor già presenti sul mercato. In seguito, saranno presentate le fasi di analisi e progettazione, dove verranno analizzati i requisiti e costruite le basi per le fasi successive. Verrà poi spiegato lo sviluppo, attraverso le tecnologie utilizzate, e l'implementazione del chatbot, seguita da un'analisi dei problemi e dell'interfaccia utente. Infine, verrà presentata una breve fase di valutazione in cui, attraverso degli usability testing, ho potuto comprendere le potenzialità e i miglioramenti necessari per il futuro.  