\chapter{Conclusioni}
\label{cha:conclusioni}

In quest'ultimo capitolo commenterò gli obiettivi raggiunti e le funzionalità implementate. Parlerò del percorso di formazione che mi ha aiutato nella progettazione e creazione del bot, dello stato attuale del lavoro e, infine, alcune considerazioni soggettive. 

\section{Raggiungimento degli obiettivi}

Gli obiettivi prefissati per questo progetto sono stati raggiunti. Tutte le fasi necessarie alla realizzazione del bot sono state eseguite. In particolare, sono partito da una prima fase di analisi dei requisiti, per poi passare alla progettazione tecnica dei vari aspetti del bot e, infine, allo sviluppo vero e proprio. 

\subsection{Funzionalità del bot}

Come descritto nel capitolo 1, le funzionalità sviluppate sono: 
\begin{itemize}
\item consultare gli orari di autobus urbani e extraurbani;
\item consultare gli orari delle ferrovie Brennero, Valsugana e Trento-Mezzana;
\item visualizzare su una mappa dove si trova un autobus in tempo reale e avere la possibilità di seguirlo durante il suo percorso;
\item visualizzare il ritardo e l'ultima posizione conosciuta dei mezzi;
\item visualizzare il numero di posti liberi nei parcheggi dei comuni di Trento e Rovereto;
\item sapere la disponibilità di bici elettriche presenti nelle stazioni di ricarica;
\item salvare le fermate e linee preferite per una consultazione di orari e ritardi molto più veloce;
\item rimanere informati su eventuali scioperi, variazioni di percorso, fermate sospese.
\end{itemize}

\subsection{Stato attuale del lavoro e sviluppi futuri}

Al momento attuale, il bot risulta funzionante. Dalla sua pubblicazione è stato utilizzato da circa 500 utenti. 

In seguito all'approfondimento dei concetti di \textit{smart city} e \textit{smart mobility} e all'analisi di alcune applicazioni facenti parte di questi ambiti, risulta chiara la direzione verso cui evolvere il bot. In particolare, i nuovi obiettivi fissati riguardano: 
\begin{itemize}
    \item una soluzione \textit{MaaS}; 
    \item i pagamenti direttamente dal bot; 
    \item il miglioramento del monitoraggio dei parcheggi; 
    \item l'espansione del bot ad altri territori, nazionali e non. 
\end{itemize}

\section{Formazione}

Per la creazione di questo bot, lo studio delle tecnologie è stato individuale. Per la fase di analisi alcuni corsi universitari, come \textit{Ingegneria del software 1 e 2} e \textit{Human-Computer Interaction}, sono risultati fondamentali per la buona riuscita. 
Grazie all'esperienza di tirocinio formativo,  presso l'azienda \textit{Zupit}, dove ho potuto approfondire varie caratteristiche dello sviluppo web, è stato più semplice e veloce imparare nuovi linguaggi e tecnologie.

\section{Conclusioni soggettive}

Il lavoro di progettazione e implementazione mi hanno dato l'opportunità di migliorare ulteriormente la preparazione offerta dal corso di studi. Lo sviluppo di un progetto pratico sarà sicuramente utile ai fini dei futuri impegni lavorativi e come aggiunta al proprio portfolio personale. 

Grazie all'esperienza di tirocinio, ho potuto seguire per intero il processo di realizzazione di alcuni progetti, questo mi ha aiutato a migliorare la gestione delle possibili problematiche legate a ciascuna fase del mio progetto, con maggiore attenzione e prontezza. 